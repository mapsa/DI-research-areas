\section{Biomedical Image Processing}
%research description here
%Some references
Image processing and computer vision techniques have evolved into valuable
tools for faster and improved interpretation of biomedical images.
We are developing computer algorithms based on machine learning, variational methods, and topological techniques for the analysis of digitized histological samples.

Specifically, we are analyzing immunohistochemistry-stained tissue images
(IHC-images) obtained through whole-slide imaging (WSI) scanners, in order to
dentify and quantify specific marker proteins.
WSI scanners are fundamental for the acquisition of high
quality and resolution tissue images, and thanks to the collaboration between
SCIAN Lab (Medicine Faculty, University of Chile) and CCTVal-UTFSM (Centro
Cient\'{i}fico Tencnol\'{o}gico de Valpara\'{i}so), we will have
access to a WSI scanner,  which will allow us to improve the image processing
algorithms in digital pathology area.

In addition, our group is working on High Performance Computing (HPC). 
The extraction of complex features from large datasets of images,
as well as training machine learning algorithms are highly demanding in terms of
computation time, storage capacity and network bandwidth. Therefore the use of
HPC is fundamental for the handling of biomedical big
data and the large scale medical image computing. 

\subsection{Publications}

Some publications made in this are are:
\begin{itemize}
\item \cite{pezoa2014}
\item \cite{pezoa2012}
\item \cite{pezoaGold}
\item \cite{pezoa2011b}
\item \cite{pezoa2011a}


\end{itemize}

