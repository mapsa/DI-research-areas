\section{Computational Astronomy}

%research description here
Astronomy has always been a data-driven science, and therefore,
the digital revolution has completely change the astronomical practice.
Moreover, current Big Data trends are challenging the classical 
data reduction process, moving out processing from the observatories and laptops, 
into large data sharing and collaboration centers. Astronomical data will
strongly increase both in size and in quantity on the next decade due to various
reasons, including the building of large projects like ALMA and E-ELT,
the improvement and deployment of new instruments, and the execution of
large astronomical surveys like the LSST project. 
As these projects (and several others) are being built in Chili, 
the opportunities for astroinformatics research in our country are vast.

Our Laboratory of Interdisciplinary Research in Astroengineering (LIRAE)
has conducted research in several astronomy-related topics during the past 
decade, for example:
\begin{itemize}
\item \textbf{Distributed telescope control}: high level control of astronomical
facilities, coordinating distributed data acquisition, real-time constraints, storage, 
federation of telescopes, alarm systems, etc. Highlighted publications are
\cite{araya08:_acs_rt} and \cite{tobar08:_csat_gtcs}.
\item \textbf{Advanced scheluling of astronomical proposals:} research on novel
algorithms in scheduling many-to-many proposals vs telescopes/antennas, with
multi-objective functions and soft and hards constraints. Highlighted
publications are \cite{mmora10:_survey_aia_dpm} and
\cite{mmora11:_solution_aia_dpm}.
\item \textbf{Software development for astronomy:} research on bleeding-edge
technologies for software development in observatories. Highlighted publications
are \cite{cmaureir10:_hpc_data_trending} and \cite{ntroncoso10:_code_generation}
\item \textbf{Virtual observatory technologies:} lead development of the
Chilean Virtual Observatory (ChiVO), including web-service and cloud technologies, 
data center standards and protocols, Big Data 
management and high-performance computing, etc. Highlighted publications are
\cite{msolar2014:_chivo_manag} and \cite{jantogni2014:_nosql}.
\item \textbf{Machine learning for astronomy:} data reduction 
and intelligent analysis of of Big Data astronomical data such as
large spectral data cubes. Highlighted publications are
\cite{rgregori:_classification_spie} and \cite{maray2014:_asydo}
\end{itemize}



%Some references
