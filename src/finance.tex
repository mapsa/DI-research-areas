\section{Computational Finances}

Since in finance a huge amount of information is required to be processed,
forecasting methods need to be fast enought in order to get quick responses.
Therefore, we are interested in financial time series forecasting using machine
learning algorithms such as online learning methods which allow to reduce
execution times by taking account only relevant past data and updating the model
by aggregating new information instead of calculate everything from scratch. 

One interesting characteristic of financial time series is that it is common to
find variables with a long-run equilibrium relationship. This relationship is
called cointegration. 

Cointegration concept was introduced by Engle, 1987 and implies that one or more
linear combinations of non-stationary variables are stationary even though
individually they are not.  Moreover, it was observed that cointegration
reflects the common stochastic trends providing a useful way to understand
cointegration relationships. This common stochastic trends can be also
interpreted as a long-run equilibrium relationships.

This idea was inmediatly adopted in finance since it could represent their
long-run relationship implied by economic theory.  Economic theory suggest that
economic time series are mean-reverting process and therefore, it reflects the
idea of that some set of variables cannot wander too far from each other. 

On the other hand, the efficient markets hypothesis, also known as the random
walk theory states that current stock prices fully reflect available information
related to its value and there is no way to earn excess profits.  This means
that if we have stock prices from a jointly efficient market, they cannot be
cointegrated. However, it was claimed that cointegration is directly at odds
with market efficiency, even though, there is no evidence that cointegration
among asset prices have implications about market efficiency.

Despite the fact that cointegration on closing daily rates of currency pairs has
not been found different time series frequencies can have different behaviours.
Pair trading is a very common example of cointegration application but
cointegration can also be extended to a larger set of variables.

Vector error correction model (VECM) introduces this long-run relationship among
a set of cointegrated variables as an error correction term. VECM is a special
case of the vector autorregresive model (VAR) model. VAR model expresses future
values as a linear combination of variables past values.  However, VAR model
cannot be used with non-stationary variables. VECM is a linear model but in
terms of variable differences. If cointegration exists, variable differences are
stationary and they introduce an error correction term which adjusts
coefficients to bring the variables back to equilibrium. In finance, many
economic time series turn to be stationary when they are differentiated and
cointegration restrictions often improves forecasting. Therefore, VECM has been
widely adopted.

Both VECM and VAR model parameters are obtained using ordinary least squares
(OLS) method. OLS has two main problems: is sensitive to errors on input data
and involves many calculations. The former problem is commonly solved using
Ridge Regression (RR) which introduces a regularization parameter that leads to
an unbiased estimation with better generalization capability. The second problem
of computational complexity depends on the number of past values and
observations considered.  Recently, online learning algorithms have been
proposed to solve problems with large data sets because of their simplicity and
their ability to update the model when new data is available. 

We are focused in providing an online formulation of the VECM which considers
only relevant past data and reducing execution time by optimizing expensive
routines, for instance: Johansen method, inverse of a matrix and OLS.  Our main
topics of interests related to this area are:

\begin{itemize}
\item High frequency trading
\item Time series and volatility forecasting
\item Online machine learning algorithms
\item Econometric models: VAR, VECM
\item High performance computing
\end{itemize}

%research description here
%Some references

\subsection{Publications}
Some recent publications made in this are are: \cite{arceetAl2015},
\cite{arce+salinas2012} and \cite{arceetAl2012}.

