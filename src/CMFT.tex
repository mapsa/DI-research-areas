\section{Computational Methods and Function Theory}

 \def\lab#1{\label{#1}}

%-------------------------mathcal------------------------------------
\def\CA{{\mathcal A }}
\def\CAO{{\mathcal A}_0}
\def\CW{{\mathcal W }}
\def\CH{{\mathcal H }}
\def\CP{{\mathcal P }}
\def\CK{{\mathcal K }}
\def\CF{{\mathcal F }}
\def\CS{{\mathcal S }}
\def\CC{{\mathcal C }}
\def\CQ{{\mathcal Q }}
%\def\CD{{\mathcal D }} % it conflicts with amscd package !!!
\def\CW{{\mathcal W}}
\def\CV{{\mathcal V}}
\def\CPn{{\mathcal P}_n}
\def\CR{{\mathcal R}}
\def\CT{{\mathcal T}}
\def\CV{{\mathcal V}}
\def\WO{{\mathcal W}_{\Omega }}

%-------------------------------Bbb-----------------------------------
\def\N{{\mathbb N }}
\def\NO{{\mathbb N }_0}
\def\Z{{\mathbb Z }}
\def\Q{{\mathbb Q }}
\def\R{{\mathbb R }}
\def\C{{\mathbb C }}
\def\F{{\mathbb F }} % anillo de los n\'umeros computables 
\def\K{{\mathbb K }}
\def\U{{\mathbb U }}
\def\D{{\mathbb D }}
\def\B{{\mathbb B }}
\def\T{{\mathbb T }}

%----------------------------Zusammengesetztes------------------------


\def\proof{\noindent{\bf Proof.\ \ }}
\def\phi{\varphi}
\def\epsilon{\varepsilon}
\def\RE{\mbox{\rm Re}\,}
\def\IM{\mbox{\rm Im}\,}
\def\AZ{|z|}
\def\COG{\CC_{\Omega{_\gamma}}}
\def\COOG{\CC(\Omega_\gamma)}
\def\CU{\CC_u}
\def\CUU{\CC_u'}
\def\CHL{\CH(\Lambda)}
\def\CHO{\CH(\Omega_\gamma^*)}

\def\F{{}_2F_1}
\def\Hyp{{}_2F_1}
\def\HY{{}_2F_1}

\def\H{{\mathcal H}(\D)}
\def\Hb{{\mathcal H}(\DB)}
\def\HO{{\mathcal H}_0(\D )}
\def\Hoo#1{{\mathcal H}(\overline{\D_{#1}})}
\def\HB#1{{\mathcal H}(\D_{#1})}
\def\Ho#1{{\mathcal H}({#1})}
\def\d#1{\D_{#1}}
\def\DB{\overline{\D}}
\def\Db#1{\overline{\D_{#1}}}
\def\DBR{\overline{\D_\rho}}

\def\DV{de la Vall\'ee Poussin }

\def\zz{\overline{z}}
%------------------------------



%-------------------------mathcal

\def\A{{\mathcal A }}
\def\CW{{\mathcal W }}
\def\CH{{\mathcal H }}
\def\CHN{\CH'}
\def\CP{{\mathcal P }}
\def\CS{{\mathcal S }}
\def\Li{{\mbox{\rm Li}}}

\def\CC{{\mathcal C }}
\def\CW{{\mathcal W}}
\def\CR{{\mathcal R}}
\def\CRu{{\mathcal R}^{\rm u}}
\def\CT{{\mathcal T}}
\def\CTT{\widetilde{{\mathcal T}}}
\def\WO{{\mathcal W}_{\Omega }}



% \def\binom#1#2{{#1\choose#2}}
% \def\dbinom#1#2{{\displaystyle{#1\choose#2}}}
\def\CCH{\overline{\mbox{\rm co}}\,}
\def\CHD{{\mathcal H}(\D)}
\def\CHDB{{\mathcal H}(\DB)}
\def\CHDC#1{{\mathcal H}(\overline{\D_{#1}})}
\def\CHDO#1{{\mathcal H}(\D_{#1})}
\def\CHO{{\mathcal H}(\Omega )}

%% \newlength{\intwidth}
%% \DeclareRobustCommand{\cpvint}[2]
%%    {\mathop{%
%%       \text{%
%%         \settowidth{\intwidth}{$\int$}%
%%         \makebox[0pt][l]{\makebox[\intwidth]{$-$}}%
%%         $\int_{#1}^{#2}$}}}

\def\d#1{\D_{#1}}
\def\DB{\overline{\D}}
\def\Db#1{\overline{\D_{#1}}}
\def\G#1{\Gamma\left({#1}+1\right)}
\def\g#1{\gamma\left(#1\right)}
\def\GA#1{\Gamma'\left({#1}+1\right)}
\def\gA#1{\gamma'\left(#1\right)}
\def\P#1{\Psi\left({#1}+1\right)}
\def\p#1{\psi\left(#1\right)}
\def\PA#1{\Psi'\left({#1}+1\right)}
\def\pA#1{\psi'\left(#1\right)}
% \def\tbinom#1#2{{\textstyle{#1\choose#2}}}

%%% LEFT AND RIGHT BRACES SPANNED THROUGH SEVERAL ROWS IN TABLE %%%%%%%
\newcommand{\rightbrace}[2]% #1=number of rows #2=width
  {\multirow{#1}{#2}{%
   $\left.{\vcenter{\hsize=0pt\vrule height #1\baselineskip
    width 0pt}} \right\}$}}
\newcommand{\leftbrace}[2]% #1=number of rows #2=width
  {\multirow{#1}{#2}{%
   $\left\{ {\vcenter{\hsize=0pt\vrule height #1\baselineskip
    width 0pt}} \right.$}}
%%%%%%%%%%%%%%%%%%%%%%%%%%%%%%%%%%%%%%%%%%%%%%%%%%%%%%%%%%%%%%%%%%%%%%%
\newcommand{\bra}[2]% #1=number of rows #2=width
  {\multirow{#1}{#2}{%
   $\left[{\vcenter{\hsize=0pt\vrule height #1\baselineskip
    width 0pt}} \right.$}}
\newcommand{\ket}[2]% #1=number of rows #2=width
  {\multirow{#1}{#2}{%
   $\left.{\vcenter{\hsize=0pt\vrule height #1\baselineskip
    width 0pt}} \right]$}}
%%%%%%%%%%%%%%%%%%%%%%%%%%%%%%%%%%%%%%%%%%%%%%%%%%%%%%%%%%%%%%%%%%%%%%%


%----------------------------Neue mathematische Operatoren----------------

\def\2F1{\mathop{\rm {}_2\,F_1}\nolimits}     % GAUSS HYPERGEOMETRIC
                                              % SERIES; 02.07.2005
\def\adj{\mathop{\rm adj}\nolimits}           % MATRIZ ADJUNTA
\def\arcosh{\mathop{\rm arcosh}\nolimits}     % AREA COSENO HYPERBOLICO
\def\arctg{\mathop{\rm arctg}\nolimits}       % ARCO TANGENTE USM
                                              % O ALEMANA 
\def\Arg{\mathop{\rm Arg}\nolimits}           % PRINCIPAL ARGUMENT
                                              % DIFFERENT FROM \arg !
\def\argmax{\mathop{\rm arg\ max}\limits}     % ARGUMENT (FOR) MAXIMUM 
                                              % (VALUE); 18.07.2004
\def\argmin{\mathop{\rm arg\ min}\limits}     % ARGUMENT (FOR) MINIMUM 
                                              % (VALUE); 18.07.2004
\def\arsinh{\mathop{\rm arsinh}\nolimits}     % AREA SENO HYPERBOLICO
\def\artgh{ \mathop{\rm artgh}\nolimits}      % AREA TANGENTE HYPERBOLICO
\def\ceiling#1{\left\lceil #1 \right\rceil}   % CEILING; 03.07.2005
\def\cis{\mathop{\rm cis}\nolimits}           % COSENO I SENO
\def\col{\mathop{\rm Col}\nolimits}           % COLUMN SPACE
\def\cond{\mathop{\bf cond}\nolimits}         % CONDITION NUMBER
\def\cosec{\mathop{\rm cosec}\nolimits}       % COSECANTE
\def\cov{\mathop{\bf cov}\limits}             % COVARIANZA; 19.07.2004
\def\ctg{\mathop{\rm ctg}\nolimits}           % COTANGENTE USM O
                                              % ALEMANA 
\def\derives#1#2{\ \overset{#1}{ \underset{#2}{\Rightarrow} }\ } %
\def\DFT{\mathop{\rm DFT}\nolimits}           % DISCRETE FOURIER TRANSFORM
\def\diag{\mathop{\rm diag}\nolimits}         % DIAG MATRIX
\def\div{\mathop{\rm div}\nolimits}           % DIVERGENCIA
\def\ds{\displaystyle}                        % DISPLAYSTYLE
\def\float{\mathop{\rm float}\nolimits}       % FLOATING OPERATOR
\def\floor#1{\left\lfloor #1 \right\rfloor}   % FLOOR; 03.07.2005
\def\grad{\mathop{\bf grad}\nolimits}         % GRADIENTE
\def\inf{\mathop{\rm inf}\limits}             % INFIMUM
\def\laplacian{\mathop{\bigtriangleup}\nolimits}  % LAPLACIANO
\def\linspan{\mathop{\rm lin\,span}\nolimits} % LINEAR SPAN; 06.08.06
\def\max{\mathop{\rm max}\nolimits}           % MAXIMUM
\def\mat#1#2#3{{\rm Mat}(#1\times #2,#3)}     % MATRIX
\def\mean{\mathcal{E}}                        % MEAN; 23.09.2005
\def\mid{\mathop{\;:\;}\nolimits}             % MID : FOR SET DEF.
\def\min{\mathop{\rm min}\nolimits}           % MINIMUM
\def\myenddemo{\hfill$\square$}               % BEWEISENDESYMBOL
% \def\nsqmat#1#2#3{M(#1\times #2,#3)}        % OLD DEF: RECTANGULAR
                                              % MATRIX
\def\nullspace{\mathop{\rm Null}\nolimits}    % RANGE SPACE
\def\range{\mathop{\rm Range}\nolimits}       % RANGE SPACE
\def\rank{\mathop{\rm rank}\nolimits}         % RANK
\def\rest#1{\mbox{\rm rest}\left\{#1\right\}} % RESTO DE DIVISION
\def\rot{\mathop{\bf rot}\nolimits}           % ROTOR
\def\seq#1#2{\ds\left\{ #1\right\}_{#2}}      % SEQUENCE; 09.05.2003
\def\sign{\mathop{\rm sign}\nolimits}         % SIGNUM
% \def\sqmat#1#2{M(#1\times #1,#2)}           % OLD DEF: SQUARE MATRIX
\def\sup{\mathop{\rm sup}\limits}             % SUPREMUM
\def\tg{\mathop{\rm tg}\nolimits}             % TANGENTE USM O ALEMANA
\def\tr{\mathop{\bf trace}\nolimits}          % TRACE
\def\var{\mathcal{V}}                         % VARIANZA; 23.09.2005
\def\ZT#1{\ds\mathfrak{Z}\left[#1\right]}     % Z-TRANSFORM; 09.05.2003
\def\ZTinv#1{\ds\mathfrak{Z}^{-1}\left[#1\right]} % INVERSE Z-TRANSFORM


%--------------% Neue Textoperatoren-----------------------------------

\def\bfref#1{{\bf\ref{#1}}}                   % BOLD FACE REF
\def\EW{valor propio}                         % EIGENWERT
\def\EWS{valores propios}                     % EIGENWERTE
\def\EV{vector propio}                        % EIGENVEKTOR
\def\EVS{vectores propios}                    % EIGENVEKTOREN

%% Gradshteyn and Ryzhik's Table of Integrals, Series, and Products
%% Alan Jeffrey and Daniel Zwillinger (eds.) Seventh edition (Feb 2007)
%% 1171 pages, ISBN number: 0-12-373637-4 (includes a CD-ROM)

%----------------------------End 0NEWMATH.STY--------------------------




%research description here
%Some references

This research area has to do with Computational Methods applied to Pure
Mathematics, in particular to Function Theory and Number Theory.

The main researchers in this area are Stephan Ruscheweyh (U. W\"urzburg,
Germany) and Luis Salinas (UTFSM, Chile).

A constant theme in these researches is convolution theory, which we briefly
recall.
For two functions $f$ and $g$ analytic in the discs
$D_{R_1}:=\{|z|<R_1\}\subset\C$ and $D_{R_2}:=\{|z|<R_2\}\subset\C$
respectively, and represented by their power series expansions
$f(z)=\sum_{k=0}^\infty a_k z^k$ and $g(z)=\sum_{k=0}^\infty b_k z^k$,
the function $f*g$ is defined by
$$
(f*g)(z) = \sum_{k=0}^\infty a_k b_k z^k\,.
$$
This function is analytic in
$D_{R_1\,R_2}:=\{|z|<R_1\,R_2\}\subset\C$
and is called {\em Hadamard product} of $f$ and $g$.
%% in honor of J. Hadamard's theorem about the location of the singularities
%% of $f*g$ in terms of the singularities of the ``factors''.
An alternative representation is by means the convolution integral
$$
(f*g)(z) = \frac{1}{2\pi i}\int_{|\zeta|=\rho}
f(z/\zeta)\,g(\zeta)\,\frac{d\zeta}{\zeta}\,,\qquad
\frac{|z|}{R_1}<\rho<R_2\,.
$$
For this reason $f*g$ is also called the {\em convolution\/} of $f$ and $g$.


\subsection{Prestarlike functions}
Let $\CH(\Omega)$ denote the set of analytic functions in a domain
$\Omega$, For domains $\Omega$ containing the origin $\CH_0(\Omega)$
stands for the set of functions $f\in\CH(\Omega)$ with 
$f(0)=1$. We also use the notation
$\CH_1(\Omega):=\{zf\,:\,f\in\CH_0(\Omega)\}$. In the special case that
$\Omega$ is the unit disc $\D:=\{z\in\C\,:\,|z|<1\}$ we use
the abbreviations $\CH,\CH_0,\CH_1$, respectively.

A function $f\in\CH_1$ is called {\em starlike of order  $\alpha$}
(with $\alpha<1$) if
$$
\RE\frac{zf'(z)}{f(z)}\geq\alpha,\quad z\in\D,
$$
and the set of such functions is denoted by $\CS_\alpha$. Then,
finally, a function $f\in\CH_1$ is called {\em prestarlike of order $\alpha$ }
if
\begin{equation}\lab{eq:1.1}
\frac{z}{(1-z)^{2-2\alpha}}*f(z)\ \in\ \CS_\alpha,
\end{equation}
The sets of these functions are denoted by $\CR_\alpha$.
One also introduces the set $\CR_1$ to consist of the functions $f\in\CH_1$
with
$$
\RE\frac{f(z)}{z}\geq\frac{1}{2},\quad z\in\D.
$$
Prestarlike functions have a number of interesting geometric
properties. For instance, the set $\CC$ of univalent functions in $\CH_1$
which map $\D$ onto convex domains equals $\CR_0$, and 
obviously we  also have $\CR_{1/2}=\CS_{1/2}$. We refer to 
%% Ruscheweyh
\cite{ru-montreal} 
and %% Sheil-Small 
\cite{ssm-book} for a description of
the essentials of the theory of prestarlike functions. 

\subsection{Universally prestarlike functions}
One of the basic goals in this research area, as far as geometric function
theory is concerned, is the translation of the notion of prestarlike functions
from the unit disc to other discs and half-planes containing the
origin.

To this end we introduced in \cite{RSS} the notion of
{\em universally prestarlike functions\/}, which we now explain.
Let $\Omega$ be a disc or half-plane in $\C$. 
Then, there are two unique parameters $\gamma\in\C\setminus\{0\}$ and 
$\rho\in[0,1]$ such that
$$
\Omega=
\left\{w_{\gamma,\rho}(z)\,:\, z\in\D\right\}\,,\qquad\text{where}\quad
w_{\gamma,\rho}(z):= \frac{\gamma z}{1-\rho z}\;.
$$
To make this relation visible, we also write $\Omega_{\gamma,\rho}$ for
this $\Omega$.

\vspace{2ex}\noindent {\bf Definition.}\ 
Let $\alpha\leq1$ and $\Omega=\Omega_{\gamma,\rho}$ for some
admissible pair $(\gamma,\rho)$.
A function $f\in\CH_1(\Omega)$ is  called {\em prestarlike of order
$\alpha$ in $\Omega$ } if 
$$
f_{\gamma,\rho}(z):=\frac{1}{\gamma}f(w_{\gamma,\rho}(z))\in\CR_\alpha.
$$
We deal with such functions specifically, and also with functions being
prestarlike of a given order in several sets $\Omega_{\gamma,\rho}$
simultaneously. 
Of course, it makes no sense to ask for functions which are prestarlike in
{\em all } such sets, because then we are left  with only the identity
function.
The situation changes already dramatically, if we admit exactly those
$\Omega_{\gamma,\rho}$ which omit one given point, for instance the point $1$. 
Note that $1\not\in\Omega_{\gamma,\rho}$ if and only if $|\gamma+\rho|\leq1$ 

A sequence $\{a_k\}_{k\geq0}$ of on-negative real numbers with $a_0=1$
is called {\em completly monotone sequences\/} (c.m.) if:
\begin{equation}
\Delta^0a_k=a\_k\geq0\,,\qquad 
\Delta^na_k := \Delta^{n-1}a_k - \Delta^{n-1}a_{k+1}\geq0\,,\quad
k\geq0\,,\quad n\geq1\,.
\end{equation}
By a well-known result of Hausdorff, $\{a_k\}_{k\geq0}$ is c.m. iff there
is a probability measure $\mu$ on the interval $[0,1]$ such that:
$a_k=\int_0^1t^k\,d\mu(t)$\,, $k\geq0$\,,
or, equivalently,
\begin{equation}
F(z):=\sum_{k=0}^{\infty} a_k\,z^k = \int_0^1 \frac{d\mu(t)}{1-tz}\,.
\end{equation}
The class of such functions $F$ is denoted by $\mathcal{T}$.

Let $\alpha\leq1$ and $\Lambda:=\C\setminus[1,\infty)$.
A function $f\in\CH_1(\Lambda)$ is called
{\em universally prestarlike of order $\alpha$\/} if and only if $f$
is prestarlike of order $\alpha$ in all sets $\Omega_{\gamma,\rho}$ with
$|\gamma+\rho|\leq1$.
The set of these functions is denoted by $\CRu_\alpha$.
This class of functions is closely related to $\mathcal{T}$.

In \cite{RS}, \cite{RSS}, \cite{barusa} we started developing the
theory of the function class $\CRu_\alpha$.
This is a currently ongoing research.

Approximation theory is one of the branches of mathematics most widely
used in engineering applications.
In this research area we recently considered a special approximation property
of the famous Riemann's function:
\begin{equation}
\zeta(s) = \sum_{n=1}^\infty \frac{1}{n^s}
= \prod_{\substack{ p\geq 2 \\ \text{$p$ prime} }}
  \left( 1 - \frac{1}{p^s} \right)\,, \qquad
s\in\C\ \text{with}\ \RE(s)>1\,.
\end{equation}
It is well known that $\zeta(s)$ has an analytic continuation to all
of $\C$ as a meromorphic function.
S.M. Voronin's Universality Theorem \cite{voronin} for Riemann's zeta function
essentially states that every non-vanishing analytic function $f(s)$ defined
on a compact subset $K$ of the critical strip
$\{s\in\C \mid 1/2 < \RE(s) < 1\}$
can be uniformly approximated by certain purely imaginary shifts of the
zeta-function. 
In \cite{herplasa} we began to study this phenomenon from the point of view
of high performance computing.
